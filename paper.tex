\documentclass[english, apj]{emulateapj}
\usepackage[T1]{fontenc}
\usepackage[latin9]{inputenc}
\usepackage{array}
\usepackage{rotating}
\usepackage{units}
\usepackage{textcomp}
\usepackage{amsmath}
\usepackage{amsbsy}
\usepackage{amstext}
\usepackage{graphicx}
\usepackage{url}
\usepackage{babel}
\usepackage{float}
\usepackage{savesym}
\savesymbol{tablenum}
\usepackage{siunitx}
\restoresymbol{SIX}{tablenum}
%\sisetup{output-exponent-marker=\ensuremath{\mathrm{e}}}
\usepackage[backref,breaklinks,colorlinks,citecolor=blue]{hyperref}
\usepackage[all]{hypcap}

\makeatletter

\newcommand{\msun}{\mbox{$\,{\rm M}_\odot$}}

\sloppy

\providecommand{\tabularnewline}{\\}
% Alter some LaTeX defaults for better treatment of figures:
    % See p.105 of "TeX Unbound" for suggested values.
    % See pp. 199-200 of Lamport's "LaTeX" book for details.
    %   General parameters, for ALL pages:
    \renewcommand{\topfraction}{0.9}	% max fraction of floats at top
    \renewcommand{\bottomfraction}{0.8}	% max fraction of floats at bottom
    %   Parameters for TEXT pages (not float pages):
    \setcounter{topnumber}{2}
    \setcounter{bottomnumber}{2}
    \setcounter{totalnumber}{4}     % 2 may work better
    \setcounter{dbltopnumber}{2}    % for 2-column pages
    \renewcommand{\dbltopfraction}{0.9}	% fit big float above 2-col. text
    \renewcommand{\textfraction}{0.07}	% allow minimal text w. figs
    %   Parameters for FLOAT pages (not text pages):
    \renewcommand{\floatpagefraction}{0.7}	% require fuller float pages
	% N.B.: floatpagefraction MUST be less than topfraction !!
    \renewcommand{\dblfloatpagefraction}{0.7}	% require fuller float pages

	% remember to use [htp] or [htpb] for placement

\begin{document}

\title{Cannibalism among the supermassive black holes}


\author{Charles Zivancev\altaffilmark{1}, Jeremiah P. Ostriker\altaffilmark{1,3}, Andreas H.W. K\"upper\altaffilmark{1,2}}
\altaffiltext{1}{Department of Astronomy, Columbia University, 550 West 120th Street, New York, NY 10027, USA}
\altaffiltext{2}{Hubble Fellow}
\altaffiltext{3}{Princeton University, New Jersey, USA}
\email{Correspondence to: csz2104@columbia.edu}




\begin{abstract}

\end{abstract}


\keywords{Galaxy: kinematics and dynamics}




\section{Introduction}\label{sec:introduction}
What happens to the massive black holes centrally located in galaxies when those galaxies merge with others?  There are three possibilities: they merge, they are ejected, or they remain in orbit in the merged galaxy.  Combinations of these options are also possible with the spin induced kick at merger leading to an ejection or extended orbit of the merged black hole.  We now know from pulsar timing measurements (ADD REFERENCES) that the gravitational wave background is probably too low for most galaxy mergers (mass-weighted) (but see \cite{2018NatCo...9..573M}) to lead to BH mergers, but the outcome remains uncertain.  We will argue in this paper that a very common outcome--especially for the lower mass black holes--is for the injected objects to remain as orbiting X-ray sources in normal massive galaxies waiting to be identified by current observational techniques.

How common are mergers of massive black holes?  For the galaxies themselves we know that minor mergers are frequent from the observed (\citet{2010ApJ...718L..73V}, \citet{2008ApJ...677L...5V}) evolution of the size and mass of these systems.  But assuming that all galaxy mergers lead to SMBH mergers overpredicts the observations of gravitational waves from the pulsar timing arrays (PTAs) (\citet{2008MNRAS.390..192S}, \citet{2009MNRAS.394.2255S}, \citet{2013MNRAS.433L...1S}, \citet{2014ApJ...789..156M}, \citet{2015ApJ...799..178K}, \citet{2018ApJ...856...42S}, \citet{2018arXiv180403143I}).  But we do not often see multiple SMBHs at the centers of massive ellipticals.  So, what does happen?  One paper (\citet{2018MNRAS.473.3410R}) has indicated that dynamical interactions among multiple black holes, which eject a non-negligible fraction of the mass, may solve this problem.  The present paper also addresses this purported dynamical solution, focusing attention on the large fraction of lower mass black holes that remain to be detected as they orbit in massive galaxies.  What we argue in this paper is that as a result of the N-body interactions among the infalling black holes some are ejected, some remain in extended orbits and some (the few most massive ones) do in fact merge, but these do not exceed the PTA limits, and the orbiting ones should be detectable via their ability to accrete gas and emit radiation.

Here we look at the merger history of 13 exemplary galaxies across the galaxy mass spectrum extracted from a cosmological simulation of hierarchical structure formation. We investigate how, after merging with incoming galaxies, SMBHs sink into the cores of the hosts and interact with the resident black hole. We show that gravitational interactions of multiple SMBHs are most probable in high-mass galaxies with total mass $10^{12} \msun < M < 10^{13} \msun$. Galaxies with lower masses have too few mergers with SMBH hosting galaxies. Galaxies with higher masses are more extended, making dynamical friction processes less efficient and hence failing to drive SMBHs into the host galaxy core.

This paper is organized as follows: in Section \ref{sec:methods} we describe the cosmological simulations from which we use the merger history to set up our idealized numerical simulations. We present the few-body integration code, \textsc{AR-Chain} that we used for our simulations of SMBH dynamics, and the modifications we made to this code in order to deal with a host galaxy's gravitational potential. In Section \ref{sec:results}, we show the results of our XX exemplary simulations of galaxies growing with time and acquiring new SMBHs. We analyze how the SMBHs are driven into the core of their new host galaxies and how interaction with the host black hole leads to near-ejections or mergers. The final Section \ref{sec:conclusions} contains a discussion of the results and our conclusions.


\section{Methods}\label{sec:methods}
%\begin{figure*}[htbp]
%\begin{center}
%\includegraphics[width=0.45\textwidth]{plots/Masses_plot_galaxy_1.png}
%\includegraphics[width=0.45\textwidth]{plots/Masses_plot_galaxy_65.png}\\
%\includegraphics[width=0.45\textwidth]{plots/Masses_plot_galaxy_187.png}
%\includegraphics[width=0.45\textwidth]{plots/Masses_plot_galaxy_217.png}\\
%\caption{default}
%\label{default1}
%\end{center}
%\end{figure*}

\subsection{Overview of simulation}
Our simulations focus on elliptical galaxies with central SMBHs. The galaxies were given a dark matter background potential using the Stone-Ostriker profile (\citet{2015ApJ...806L..28S}), which is a three-parameter potential-density pair, whose quantities such as density, potential, and binding energy can be written in closed form, having a profile similar to that of a truncated isothermal sphere.  The galaxies were evolved from $4 > z > 0$.  Orbiting black holes were periodically introduced into the "host" galaxy following mergers and their dynamical interaction with the background potential and central SMBH were followed, as described further below.

For the numerical simulations presented here, we used a modified version of the algorithmic chain integrator \textsc{AR-Chain} developed by \citet{2006MNRAS.372..219M}. It uses algorithmic chain regularization for high-precision integration of few-body dynamics, and is capable of handling velocity-dependent forces efficiently. It includes relativistic post-Newtonian terms up to order PN2.5 \citep{2008AJ....135.2398M}.


\subsection{Merger tree Data}
The merger tree data that was used as the input to our simulations is the result of previous work by \citet{2011ApJ...741...99C, 2011ApJ...742L..33C, 2012ApJ...753...17C, 2012ApJ...748..121C, 2013ApJ...770..139C} (hereafter Cen11), \citet{2012MNRAS.425..641L} (hereafter Lackner12), and \citet{2015ApJ...799..178K} (hereafter Kulier15).  A brief synopsis of the work that led to our input data is described below.

Cen11 studied the "cosmic downsizing" effect (e.g., \citet{1996AJ....112..839C}) using high-resolution large-scale hydrodynamic galaxy formation simulations.  His work produced physical parameters for galaxies such as position, velocity, total mass, stellar mass, gas mass, mean formation time, mean stellar metallicity, mean gas metallicity, SFR, luminosities, etc.

Lackner12, using Cen11's work as a basis, created galactic merger trees that compared the properties of in situ and accreted stellar mass of galaxies at redshift snapshots between $4 > z > 0$.

Kulier15 furthered the merger tree data produced by Lackner12 by estimating the evolution of the SMBH population, based on assumptions about the relationship between the properties of SMBHs and those of their host galaxies.

The primary data we extracted from the above work was organized as follows:
\begin{itemize}
    \item Galactic properties - Each galaxy was distinguished by an id number, with properties such as stellar mass, dark matter mass, central black hole id number, and orbiting black hole id numbers, recorded at 37 redshift slices between $4 > z > 0$.
    \item Black Hole properties - Each black hole was also distinguished by an id number, with properties such as seed mass, accreted mass, its host galaxy id number, and time after $z=4$ at which the black hole entered its host galaxy, if it was not the central black hole.
\end{itemize}

The merger tree data from Lackner12 and Kulier15 provided 1,830 galaxies in total.  However, not all the galaxies were suitable for our simulations.  We placed further requirements as follows:
\begin{itemize}
\item The galaxies had to exist through the entire simulation ($4 > z > 0$).  If they merged with other galaxies, they had to have been the "surviving" galaxy at each merger.
\item They had to have accumulated orbiting black holes by $z = 0$.
\end{itemize}

The above criteria narrowed the field of eligible galaxies down to 51.  In considering time and resources, we chose the top 13 galaxies in terms of number of black holes to run the simulations.  The black holes in these galaxies accounted for nearly $84{\%}$ of all orbiting black holes across all the galaxies, so it represented a reasonable balance between completeness and efficiency.

The following cosmological parameters were used in our simulations, consistent with both Lackner12 and Kulier15:   $\Omega_M = 0.28$, $\Omega_b = 0.046$, $\Omega_\Lambda = 0.72$, $\sigma_8 = 0.82$, $H_0 = 100h^{-1}Mpc^{-1} = 70 km s^{-1} Mpc^{-1}$, and $n = 0.96$.

\subsubsection{Stellar Mass Adjustment}
As we will explain further in Section \ref{Galaxy background potential}, in our simulations orbiting black holes are inserted into the host galaxy at the effective radius, $R_e$, which is partially dependent on stellar mass, $M_*$ (see equation \ref{re}).  For every doubling of the stellar mass, $M_*$, $R_e$ increases by a factor of 1.66.  Additionally, the velocity dispersion at the effective radius, $\sigma(R_e)$, which is used to set the parameter $r_h$ of our dark matter profile, is also dependent on $M_*$.  In this case, a doubling of $M_*$ would increase $\sigma(R_e)$ by a factor of 1.15.  These variations have direct effects on the ability of the orbiting black holes to merge with the central black hole, and thus can affect our merger statistics and resulting characteristic strain calculations, and therefore it is important to estimate our stellar masses with reasonable accuracy.

In cosmological simulations it is common to overproduce stellar mass, with a general over-efficiency in the range of 2-4 times those measured in observations (\citet{1996ApJS..105...19K}, \citet{2010MNRAS.404.1111G}, \citet{2010ApJ...725.2312O}).  One typical reason cited for this overproduction is a lack or underestimation of supernova feedback effects in the form of thermal and/or kinetic energy.  However, even when thermal energy from supernova feedback is considered, the effects may be reduced due to the energy being quickly radiated away by rapidly cooling surrounding gas (\citet{1996ApJS..105...19K}).  A second common reason for overproduction of stars in cosmological simulations is a lack of modeling for AGN feedback, which may be responsible for a phenomenon called the "cooling flow paradox" (\citet{2001MNRAS.321L..20F}) that suppresses star formation. 

Lackner12, which is the source of our stellar masses, noted that the efficiency of star formation in their simulations, defined as $f_*=M_*/M_{DM}(\Omega_{DM}/\Omega_b)$, was approximately 0.6.  Compared with the expected range of $0.10 \lessapprox f_* \lessapprox 0.15$ that they referenced from \citet{2012ApJ...746...95L}, their stellar masses were a factor of roughly 4 times greater.

We also compared the stellar masses from Lackner12 to  observational and abundance matching data from \citet{2018AstL...44....8K}. Figure \ref{fig:stellar1} is a partial, approximate, replication of their Figure 11 (abundance matching line).  We rescaled our stellar masses at $z=0$ in a lognormal manner about the line in order to match the observed data.  Table \ref{table:gal_char} shows the original and rescaled stellar masses.

\begin{figure}[htbp]
\begin{center}
\includegraphics[width=1.0\columnwidth]{plots/stellar_to_halo_ratio.png}
\caption{Approximation of Figure 11 from \citet{2018AstL...44....8K}.  Purple dots shown is example of our rescaled stellar masses}
\label{fig:stellar1}
\end{center}
\end{figure}

\begin{table}[htbp]
\caption{Galaxy characteristics at z=0}
\begin{tabular} {|c | c | c| c|}
\hline
$M_{DM}$ & $M_{CBH}$ & $M_{*} (orig.)$ & $M_{*} (rescaled)$ \\
\hline
 \num{9.18e13} 	&	 \num{2.19e9} 	&	\num{1.75e13} 	&	\num{5.14e11}  \\
 \num{7.75e13} 	&	 \num{5.27e9}	&	\num{1.52e13} 	&	\num{4.00e11}  \\
 \num{2.85e13} 	&	 \num{1.46e9}	&	\num{5.58e12} 	&	\num{3.30e11}  \\
 \num{2.48e13} 	&	 \num{2.40e9}	&	\num{4.80e12} 	&	\num{2.25e11}  \\
 \num{2.29e13} 	&	 \num{1.22e9}	&	\num{4.46e12} 	&	\num{2.68e11}  \\
 \num{1.18e13} 	&	 \num{3.62e9}	&	\num{2.32e12} 	&	\num{2.83e11}  \\
 \num{1.03e13} 	&	 \num{9.85e9}	&	\num{1.96e12} 	&	\num{1.62e11}  \\
 \num{1.00e13} 	&	 \num{9.32e9}	&	\num{1.95e12} 	&	\num{2.49e11}  \\
 \num{8.92e12} 	&	 \num{5.60e9}	&	\num{1.74e12} 	&	\num{3.20e11}  \\
 \num{8.56e12} 	&	 \num{6.86e9}	&	\num{1.62e12} 	&	\num{1.83e11}  \\
 \num{6.94e12} 	&	 \num{2.02e9}	&	\num{1.72e12} 	&	\num{1.51e11}  \\
 \num{6.29e12} 	&	 \num{1.22e9}	&	\num{1.23e12} 	&	\num{1.87e11}  \\
 \num{1.84e12} 	&	 \num{2.21e9}	&	\num{5.52e11} 	&	\num{6.42e10}  \\
\hline
\end{tabular}

\label{table:gal_char}
\end{table}

\subsection{AR-Chain code}
Summary of the code and the modifications we made.

\subsubsection{Galaxy background potential} \label{Galaxy background potential}
For our simulations, the Stone-Ostriker profile (\citet{2015ApJ...806L..28S}), which is a three-parameter potential-density pair, whose quantities such as density, potential, and binding energy can be written in closed form.  It is essentially an analytic form of a finite, cored isothermal mass distribution:
\begin{equation} \label{jerry}
\rho(r) = \frac{\rho_c}{(1+r^2/r_{c}^2)(1+r^2/r_{h}^2)}
\end{equation}
Here $\rho_c$ is the central density, $r_c$ is the core radius, and $r_h$ is the outer halo radius.

In order to parametrize this profile, $r_c$ was given an initial value of 100 pc, which is reasonable for a cored massive system.  To calculate $r_h$, we began by using the stellar mass to calculate the velocity dispersion at the galaxy's effective radius.  In Kulier12, the galaxy's effective radius $R_{e}$ and velocity dispersion $\sigma(R_e)$ at the effective radius are:
\begin{equation} \label{re}
R_{e} = 2.5 kpc\left(\frac{M_*}{10^{11}M_{\odot}}\right)^{0.73}(1+z)^{-0.98}
\end{equation}
\begin{equation} \label{sig}
\sigma(R_{e}) = 190km/s\left(\frac{M_{*}}{10^{11}M_{\odot}}\right)^{0.2}(1+z)^{0.47}
\end{equation}

We can then equate the value for $\sigma({R_e})$ obtained from Equation \ref{sig} to the analytic expressions for $\sigma(R_{e})$ in the Stone-Ostriker profile (Eqns 9 and A1-A4).  The only unknown is $r_h$, which we can solve for using a simple recursive Newton method.  Whether $\sigma_{near}$ or $\sigma_{far}$ is used from Stone-Ostriker is determined by whether $R_e$ is less than or greater than $\sqrt{r_c r_h}$.

The central density, $\rho_c$, can be found from Equation 5 in Stone-Ostriker:
\begin{equation} \label{rhoc}
M_{tot} = \frac{2\pi^2r_{c}^2r_{h}^2\rho_c}{r_h+r_c},
\end{equation}
where $M_{tot}$ is the total dark matter mass.

At each iteration in the code, $r_h$ is updated according to the stellar mass and Equations \ref{re} and \ref{sig}.  The core radius, $r_c$, is recalculated only if an orbiting black hole gets within $r_c$.  The work done by diffusion, as described in Section \ref{psd} is calculated, the total potential energy and dynamical friction are updated, and $r_c$ is solved for from Equation 8 in Stone-Ostriker.

\subsubsection{Phase-space diffusion} \label{psd}
Weak encounters with background stars will let the SMBHs diffuse through phase space while they are orbiting within the gravitational potential of the galaxy. The diffusion can be expressed as change in velocity of an SMBH by $\Delta \vec{v}$ per unit time. We can split this change into a component along the direction of motion of the SMBH, and one perpendicular to that. Following \citet{2008gady.book.....B}, the diffusion coefficients can be expressed as 
\begin{eqnarray}\label{eq:df}
D[\Delta v_\parallel] & = & -\frac{4\pi G^2\rho(r)M_\bullet\ln\Lambda}{\sigma^2}f(\chi),\\
D[(\Delta v_\parallel)^2] & = & \frac{4\sqrt{2}\pi G^2\rho(r)M_\bullet\ln\Lambda}{\sigma}\frac{f(\chi)}{\chi},\\
D[(\Delta \vec{v}_\bot)^2] & = & \frac{4\sqrt{2}\pi G^2\rho(r)M_\bullet\ln\Lambda}{\sigma}\left[\frac{\mbox{erf}(\chi)-f(\chi)}{\chi}\right],
\end{eqnarray} 
where $\Delta v_\parallel \equiv \Delta \vec{v}\cdot\vec{v}/v$ is the velocity change in direction of motion, and $\Delta \vec{v}_\bot \equiv \Delta \vec{v} - \Delta v_\parallel \cdot\vec{v}/v$ is the velocity change perpendicular to the direction of motion. Here, $M_\bullet$ is the mass of the black hole, and $\chi = \frac{v}{\sqrt{2}\sigma(r)}$. The function $f(\chi)$ is given by 
\begin{equation}
f(\chi) \equiv \frac{1}{2\chi^2}\left(\mbox{erf}(\chi)-\frac{2\chi}{\sqrt{\pi}}\exp\left(-\chi^2\right)\right).
\end{equation}
We approximate the factor $\Lambda$ in the Coulomb logarithm as
\begin{equation}
\Lambda \equiv \left(\frac{M_{NSC}}{M_\bullet}\right)\left(\frac{r}{r_h}\right).
\end{equation}
We can identify Equation \ref{eq:df} as the dynamical friction term. The second term introduces a variance of the friction term, and even allows the SBHs to be accelerated when the velocity of a SBH is sufficiently small. The third term introduces a change in velocity perpendicular to the direction of motion of the SBH. It is a randomly oriented vector, and hence causes the SBHs to execute a random walk in phase space. The last two terms will establish that the SBHs are ultimately in energy equipartition with the background stars.
The velocity changes $\Delta v_\parallel$ and $\Delta\vec{v}_\bot$ per unit time $\Delta t$ can be computed with the above equations. Both changes are normally distributed, where the mean, $\mu$, and the variance, $\Sigma$, of the distributions are given by
\begin{eqnarray}
\mu_\parallel &=& D[\Delta v_\parallel]\Delta t,\\
\Sigma_\parallel &=& D[(\Delta v_\parallel)^2]\Delta t,\\
\mu_\bot &=& 0,\\
\Sigma_\bot &=& D[(\Delta \vec{v}_\bot)^2]\Delta t.
\end{eqnarray}
We compute the diffusion coefficients for each black hole at each time step, and modify its velocity on a Monte Carlo basis. For each time step we draw a random orientation before adding the perpendicular velocity change to the respective SBH. Hence, the SBH's modified velocity, $v_f$, is computed using
\begin{eqnarray}
\vec{v}_f &=& \vec{v}_0 + \Delta v_\parallel \hat{v}_\parallel + \Delta v_\bot \hat{v}_\bot,\\
\Delta v_\parallel &=& \mathcal{N}(\mu_\parallel, \Sigma_\parallel),\\
\Delta v_\bot &=& \mathcal{N}(\mu_\bot, \Sigma_\bot).
\end{eqnarray}
The change of energy, $\mbox{d}E_{BH}$, of the orbiting black hole due to phase-space diffusion is given back to the stellar background potential, with $\mbox{d}E = -\mbox{d}E_{BH}$. As a consequence of this energy transfer, inspiralling black holes will cause an expansion of the NSC. For this purpose we calculate the change in potential energy, $\mbox{d}W$, of the stellar system using
\begin{eqnarray}
E &=& T + W = \frac{1}{2}W,\\
\mbox{d}W &=& -2\,\mbox{d}E_{BH},
\end{eqnarray}
where we made use of the virial theorem $2T+W =0$. With this change in potential energy we can calculate a new radius for the stellar background potential at each integration step. For the Plummer sphere the new scale radius can be calculated as
\begin{equation}
a_{new} = a\left(1+\frac{32a\,\mbox{d}W}{3\pi GM_{NSC}^2}\right)^{-1}.
\end{equation}

\subsubsection{Gravitational wave recoils}
The code \textsc{AR-Chain} includes post-Newtonian terms up to order 2.5. The SMBHs can therefore merge via gravitational wave emission. We include gravitational wave recoils following the prescription outlined in Kulier15, which is based on the fitting formula by \citet{2012PhRvD..85h4015L}.  We assume that a merger will be inevitable when the separation between two SBHs becomes smaller than 1.0 Schwarzschild radius. At the moment of the merger, we assume that the spin vectors of the two SBHs are randomly aligned.

Black holes can also eject each other via strong three-body interactions. We remove SBHs from the simulations once they move beyond $r_h$, assuming that it will take them more than a Hubble time to find their way back into the center of the host galaxy.

\subsection{Simulation setup}
Using the merger tree data, orbiting black holes were injected into its host galaxy at redshift z, at a distance from the galactic center of $R_{e}$ (Eqn. \ref{re}) following galaxy mergers.  Their initial velocity was arbitrarily chosen to be circular, $v_c = \sqrt{M(R_e)/R_e}$, with $v_x$, $v_y$, and $v_z$ randomly chosen.  Note: we only inject the BHs into the simulation that have a $t_{fric}$ smaller than 100 times the Hubble time.

\subsection{Black Hole Luminosity}
The x-ray temperature versus sigma relation from Table 2 of  \citet{2018ApJ...857...32B} was fitted according to:
\begin{equation}
    T_x = 10^{(2.50*log \sigma - 6.06)}
\end{equation}
Central gas density, ${\rho}_0$ was also fitted from Table 3 of \citet{2018ApJ...857...32B} according to:
\begin{equation}
    \rho_0 = 10^{(0.6log(\sigma) - 25)}
\end{equation}
The core radius of the gas model, ${r_c}$ is:
\begin{equation}
    r_c = 10^{(15.09log(\sigma) - 35.77)}
\end{equation}
Local gas density ${\rho}$ is obtained from the isothermal beta model:
\begin{equation}
    \rho = \frac{\rho_0}{(1+r^2/r_c^2)^{1.5}}
\end{equation}
Gas sound speed, ${ss}_g$, was obtained from:
\begin{equation}
    {ss}_g = \sqrt{1.67T_x/m_p},
\end{equation}
where $m_p$ is the proton mass.

The black hole mass accretion rate is approximated by:
\begin{equation}
    \dot{M}_{acc} = \frac{4{\pi}G^2M^2\rho}{({ss_g}^2+v^2)^{3/2}},
\end{equation}
where M is black hole mass, G is the gravitational constant, and v is the black hole velocity.

Finally, the black hole luminosity is calculated from:
\begin{equation}
    L = 0.1\dot{M}_{acc}{ss_g}^2
\end{equation}

The Eddington limit is known to be:
\begin{equation}
    L_{edd} = \num{1.26e38} (M/M_{\odot}) erg/s
\end{equation}

\subsection{Description of Simulations}
\begin{table*}
\centering
\caption{Galaxy characteristics}
\begin{tabular}{c| c c| c c| c c}
 & \multicolumn{2}{c}{$M_{gal}$ [$M_{\odot}$]} & 
\multicolumn{2}{c}{$M_{*}$ [$M_{\odot}$]} & 
\multicolumn{2}{c}{$M_{BH}$ [$M_{\odot}$]} \\
\hline
Galaxy & Init & Final & Init & Final & Init & Final \\
 \hline
A & $7.42\times10^{11}$ & $1.09\times10^{14}$  & $1.22\times10^{11}$ & $1.75\times10^{13}$ & $4.80\times10^{6}$ & $2.19\times10^{9}$\\
B & $1.11\times10^{12}$ & $3.41\times10^{13}$ & $1.80\mathrm{3}{11}$ & $5.58\times10^{12}$ & $1.94\times10^{8}$ & $1.46\times10^{9}$\\
C & $7.23\times10^{10}$ & $3.48\times10^{12}$ & $1.18\times10^{10}$ & $5.68\times10^{11}$ & $9.47\times10^{4}$ & $2.70\times10^{8}$\\
D & $3.92\times10^{10}$ & $1.34\times10^{12}$ & $6.47\times10^{9}$ & $2.20\times10^{11}$ & $9.95\times10^{5}$ & $8.32\times10^{7}$\\
\end{tabular}
\end{table*}





\section{Results}\label{sec:results}
\begin{figure}[ht]
\begin{center}
\includegraphics[width=0.45\textwidth]{"plots/Fractions All Galaxies".png}
\caption{Fractions Merged/Ejected/Orbiting at z=0, by mass bin}
\label{fig:fmeo}
\end{center}
\end{figure}

\begin{figure}[ht]
\begin{center}
\includegraphics[width=0.45\textwidth]{plots/"Summary Results All Galaxies".png}
\caption{Histogram of merged, ejected, and orbiting SMBHs}
\label{fig:meosmbh}
\end{center}
\end{figure}

\begin{figure}[ht]
\begin{center}
\includegraphics[width=0.45\textwidth]{plots/"Final Luminosities of Oriting BHs".png}
\caption{Luminosities of Orbiting BHs at z=0}
\label{fig:loobhs}
\end{center}
\end{figure}

\section{Conclusions}\label{sec:conclusions}
What do we want to say?



\section*{Acknowledgements}

The authors would like to thank Andrea Kulier and Claire Lackner for providing merger tree data. AHWK acknowledges support by NASA through Hubble Fellowship grant HST-HF-51323.01-A awarded by the Space Telescope Science Institute, which is operated by the Association of Universities for Research in Astronomy, Inc., for NASA, under contract NAS 5-26555. 

\bibliographystyle{apj}
\bibliography{biblio}

%\begin{thebibliography}{}

%\bibitem[\protect\citeauthoryear{Aarseth}{2003}]{Aarseth03}
%Aarseth, S.~J., 2003, Gravitational N-Body Simulations (Cambridge University Press)


%\bibitem[\protect\citeauthoryear{K{\"u}pper et al.}{2011}]{Kupper11} 
%K{\"u}pper, A.~H.~W., Maschberger, T., Kroupa, P., Baumgardt, H., 2011, MNRAS, 417, 2300 

%\end{thebibliography}
\end{document}

