\documentclass[english, apj]{emulateapj}
\usepackage[T1]{fontenc}
\usepackage[latin9]{inputenc}
\usepackage{array}
\usepackage{rotating}
\usepackage{units}
\usepackage{textcomp}
\usepackage{amsmath}
\usepackage{amsbsy}
\usepackage{amstext}
\usepackage{graphicx}
\usepackage{url}
\usepackage{babel}
\usepackage[backref,breaklinks,colorlinks,citecolor=blue]{hyperref}
\usepackage[all]{hypcap}

\makeatletter

\newcommand{\msun}{\mbox{$\,{\rm M}_\odot$}}

\sloppy

\providecommand{\tabularnewline}{\\}

\begin{document}

\title{Cannibalism among the supermassive black holes}


\author{Charles Zivancev\altaffilmark{1}, Andreas H.W. K\"upper\altaffilmark{1,2}, Jeremiah P. Ostriker\altaffilmark{1,3}}
\altaffiltext{1}{Department of Astronomy, Columbia University, 550 West 120th Street, New York, NY 10027, USA}
\altaffiltext{2}{Hubble Fellow}
\altaffiltext{3}{Princeton University, New Jersey, USA}
\email{Correspondence to: csz2104@columbia.edu}




\begin{abstract}

\end{abstract}


\keywords{Galaxy: kinematics and dynamics}




\section{Introduction}\label{sec:introduction}



\section{Methods}\label{sec:methods}


\subsection{Merger tree}
Where we got the merger data from and how we extracted it.
- Using Andrea Kulier's cosmological simulation, we plotted galaxy masses as a function of time.
- We took the data from the three most massive galaxies (ie, most massive at redshift z=0).
- We curve-fit each galaxy's mass to polynomial functions of time.
- We also curve-fit each galaxy's (accreted+seeded) central black hole mass to exponential functions of time.
- We added those functions to the Fortran hermite code
- We implemented the SO potential-density pair (using the m200, r200 from the fitting functions above, getting half-mass radius from NFW profile, and using it as half-mass radius for the SO profile).

Simulation of Cen et al. (have to look up the year)


\subsection{AR-Chain code}
Summary of the code and the modifications we made.

For the numerical simulations presented here, we used a modified version of the algorithmic chain integrator \textsc{AR-Chain} developed by \citet{Mikkola06}. It uses algorithmic chain regularization for high-precision integration of few-body dynamics, and is capable of handling velocity-dependent forces efficiently. It includes relativistic post-Newtonian terms up to order PN2.5 \citep{Mikkola08}. 


\subsubsection{Galaxy background potential}

\subsubsection{Phase-space diffusion}
Weak encounters with background stars will let the SMBHs diffuse through phase space while they are orbiting within the gravitational potential of the galaxy.

\subsubsection{Gravitational wave recoils}
The code \textsc{AR-Chain} includes PN terms up to order 2.5. The SMBHs can therefore merge via gravitational wave emission. We include gravitational wave recoils following the prescription outlined in \citet{Kulier_2015}, which is based on the fitting formula by \citet{Lousto12}. To save computational time, we assume that a merger will be inevitable when the separation between two SBHs gets smaller than 10\,000 Schwarzschild radii. At the moment of the merger, we assume that the spin vectors of the two SBHs are randomly aligned. This results in kick velocities of up to several thousand km\,s$^{-1}$, with a median kick of $\approx 290$\,km\,s$^{-1}$. Since our simulations focus on NSC with relatively low escape velocities, this implies that a majority of the merging SBHs escape from the NSCs. 

Black holes can also eject each other via strong three-body interactions. We remove SBHs from the simulations once they move beyond 1\,kpc from the NSC, assuming that it will take them more than a Hubble time to find their way back into the center of the host galaxy.


\subsection{Simulation setup}
Injection, merging, escape, effective radius, set of galaxies


\section{Results}\label{sec:results}

\subsection{Mergers}

\begin{figure}
\centering
%\includegraphics[width=0.45\textwidth]{plots/plot1.png}
\caption{This is a caption.}
\label{this_is_a_label}
\end{figure}

\subsection{Ejections}

\subsection{Stalling SMBH binaries}

\subsection{Stalling SMBHs in the halo}





\section{Conclusions}\label{sec:conclusions}
What do we want to say?



\section*{Acknowledgements}

The authors would like to thank Andrea Kulier for providing merger tree data. AHWK acknowledges support by NASA through Hubble Fellowship grant HST-HF-51323.01-A awarded by the Space Telescope Science Institute, which is operated by the Association of Universities for Research in Astronomy, Inc., for NASA, under contract NAS 5-26555. 


\bibliographystyle{apj}
\begin{thebibliography}{}

\bibitem[\protect\citeauthoryear{Aarseth}{2003}]{Aarseth03}
Aarseth, S.~J., 2003, Gravitational N-Body Simulations (Cambridge University Press)

\bibitem[\protect\citeauthoryear{K{\"u}pper et al.}{2011}]{Kupper11} 
K{\"u}pper, A.~H.~W., Maschberger, T., Kroupa, P., Baumgardt, H., 2011, MNRAS, 417, 2300 

\end{thebibliography}
\end{document}

