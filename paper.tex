\documentclass[english, apj]{emulateapj}
\usepackage[T1]{fontenc}
\usepackage[latin9]{inputenc}
\usepackage{array}
\usepackage{rotating}
\usepackage{units}
\usepackage{textcomp}
\usepackage{amsmath}
\usepackage{amsbsy}
\usepackage{amstext}
\usepackage{graphicx}
\usepackage{url}
\usepackage{babel}
%\usepackage{siunitx}
%\sisetup{output-exponent-marker=\ensuremath{\mathrm{e}}}
\usepackage[backref,breaklinks,colorlinks,citecolor=blue]{hyperref}
\usepackage[all]{hypcap}

\makeatletter

\newcommand{\msun}{\mbox{$\,{\rm M}_\odot$}}

\sloppy

\providecommand{\tabularnewline}{\\}

\begin{document}

\title{Cannibalism among the supermassive black holes}


\author{Charles Zivancev\altaffilmark{1}, Jeremiah P. Ostriker\altaffilmark{1,3}, Andreas H.W. K\"upper\altaffilmark{1,2}}
\altaffiltext{1}{Department of Astronomy, Columbia University, 550 West 120th Street, New York, NY 10027, USA}
\altaffiltext{2}{Hubble Fellow}
\altaffiltext{3}{Princeton University, New Jersey, USA}
\email{Correspondence to: csz2104@columbia.edu}




\begin{abstract}

\end{abstract}


\keywords{Galaxy: kinematics and dynamics}




\section{Introduction}\label{sec:introduction}
A majority of all galaxies with masses $>10^8\msun$ at $z=0$ are believed to host super-massive black holes (SMBHs) in their cores. In a $\Lambda$CDM universe, galaxies at the high end of the galaxy-mass function are merger products of numerous such massive galaxies. Even if only a small fraction of merging galaxies hosted an SMBH in their core at the time of merging, it is reasonable to assume that, during a Hubble time of continuous growth, these galaxies will have substantial phases in which they contain two or more SMBHs at the same time. 

Here we look at the merger history of four exemplary galaxies across the galaxy mass spectrum extracted from a cosmological simulation of hierarchical structure formation. We investigate how, after merging with incoming galaxies, SMBHs diffuse into the cores of the hosts and interact with the resident black hole. We show that gravitational interactions of multiple SMBHs are most probable in high-mass galaxies with $10^{12} \msun < M < 10^{13} \msun$. Galaxies with lower masses  have too few mergers with SMBH hosting galaxies. Galaxies with higher masses are more extended, making dynamical friction processes less efficient and hence failing to drive SMBHs into the host galaxy core.  Several papers (\citet{2018arXiv180403143I}, \citet{2014ApJ...789..156M})have shown that if all galaxy mergers lead to black hole mergers, the resulting PTA signal would be roughly twice the current limits.  One paper (\citet{2018MNRAS.473.3410R}) has indicated that dynamical interactions among multiple black holes, which eject a non-negligible fraction of the mass, may solve this problem.  The present paper also addresses this purported dynamical solution.

This paper is organized as follows: in Section \ref{sec:methods} we are going to describe the cosmological simulation from which we use the merger history to set up our idealized numerical simulations. We present the few-body integration code, \textsc{AR-Chain} that we used for our simulations of SMBH dynamics, and the modifications we made to this code in order to deal with a host galaxy's gravitational potential. In Section \ref{sec:results}, we show the results of our four exemplary simulations of galaxies growing with time and acquiring new SMBHs. We analyze how the SMBHs are driven into the core of their new host galaxies and how interaction with the host black hole leads to near-ejections or mergers. The final Section \ref{sec:conclusions} contains a discussion of the result and our conclusions.


\section{Methods}\label{sec:methods}
%\begin{figure*}[htbp]
%\begin{center}
%\includegraphics[width=0.45\textwidth]{plots/Masses_plot_galaxy_1.png}
%\includegraphics[width=0.45\textwidth]{plots/Masses_plot_galaxy_65.png}\\
%\includegraphics[width=0.45\textwidth]{plots/Masses_plot_galaxy_187.png}
%\includegraphics[width=0.45\textwidth]{plots/Masses_plot_galaxy_217.png}\\
%\caption{default}
%\label{default1}
%\end{center}
%\end{figure*}

\subsection{Overview of simulation}
Our simulations focused on elliptical galaxies with central SMBHs. The galaxies were given a background potential based on the Ostriker-Stone profile (\citet{2015ApJ...806L..28S}), which is a three-parameter potential-density pair, whose quantities such as density, potential, and binding energy can be written in closed form.  The galaxies were evolved between $0 < z < 4$.  Orbiting black holes were periodically introduced into the "host" galaxy and their dynamical interaction with the background potential and central SMBH were followed.

For the numerical simulations presented here, we used a modified version of the algorithmic chain integrator \textsc{AR-Chain} developed by \citet{2006MNRAS.372..219M}. It uses algorithmic chain regularization for high-precision integration of few-body dynamics, and is capable of handling velocity-dependent forces efficiently. It includes relativistic post-Newtonian terms up to order PN2.5 \citep{2008AJ....135.2398M}.


\subsection{Merger tree}
Our merger tree data came from two sources:
\begin{itemize}
\item Galactic merger tree data from simulations of \citet{2012MNRAS.425..641L} (hereafter Lackner12), which was centered on galaxies at each redshift slice, providing information on the galaxies' stellar mass, dark matter mass, central BH, and any orbiting black holes within the galaxy.
\item SMBH evolution simulations of \citet{2015ApJ...799..178K} (hereafter Kulier15), which used the large-scale hydrodynamical galaxy simulations outlined in \citet{2011ApJ...741...99C, 2011ApJ...742L..33C, 2012ApJ...753...17C, 2012ApJ...748..121C, 2013ApJ...770..139C}.  The data was centered around SMBHs at each redshift slice between $0 < z < 4$, giving information for each SMBH's seed mass, accreted and seed mass, its host galaxy (categorically defined), the stellar mass in its host galaxy, and time after z=4 at which the BH entered the host galaxy (if it is not the central BH).
\end{itemize}

SHOULD I INCLUDE PLOTS AND/OR HISTOGRAMS OF THE DISTRIBUTION OF GALAXY MASSES, STELLAR MASSES, ETC AT Z=0?

The following cosmological parameters were used in the simulations of both Lackner12 and Kulier15:   $\Omega_M = 0.28$, $\Omega_b = 0.046$, $\Omega_\Lambda = 0.72$, $\sigma_8 = 0.82$, $H_0 = 100h^{-1}Mpc^{-1} = 70 km s^{-1} Mpc^{-1}$, and $n = 0.96$.

Lackner12 noted that the efficiency of star formation in their simulations, defined as $f_*=M_*/M_{DM}(\Omega_{DM}/\Omega_b)$, was approximately 0.6.  Compared with the expected range of $0.10 \lessapprox f_* \lessapprox 0.15$ that they referenced from \citet{2012ApJ...746...95L}, their stellar masses were a factor of roughly 4 times greater.  In order to adjust our stellar masses to levels more in line with current observations, we used observational data from \citet{2018AstL...44....8K}, in particular Figure 11. (SHOULD I ADD PLOT OF OUR ORIGINAL STELLAR DATA?  THE PLOT FROM KRAVTSOV WAS RECREATED BY EYEBALL WITH A FEW STRAIGHT LINES TO APPROXIMATE HIS CURVE.  HOW TO INCORPORATE THAT?)

Starting with this data, we plotted the galaxy masses as a function of time.  Since the raw data was given in terms of redshift, we converted redshift to time.  We chose four galaxies to run our simulations.  The galaxies were chosen using the following basic criteria:
\begin{itemize}
\item The galaxies had to exist through the entire simulation ($0 < z < 4$).
\item They had to have a central SMBH.
\item They had to have at least XXX orbiting black holes.
\item We ordered the galaxies by decreasing mass at $z=0$, and starting with the largest tier of mass (~$10^{14} M_{\odot}$), we spaced their masses by approximately 1-2 orders of magnitude.  We chose the four largest galaxies that also satisfied the above criteria.
\end{itemize}

After finding our subject galaxies, we curve-fit their total masses as a function of time using a $7^{th}$-order polynomial fit, so as to be able to use their masses in the AR-Chain code.  Additionally, we curve-fit each galaxy's accreted+seed central black hole mass to exponential functions of time.



\subsection{AR-Chain code}
Summary of the code and the modifications we made.



\subsubsection{Galaxy background potential}
The simulations from which we obtained the merger tree data considered only dark matter distributions and were run using the Navarro-Frenk-White (NFW) profile \citet{1997ApJ...490..493N}.  Since our study is more interested SMBH mergers at the center of galaxies, dynamical friction, and hence baryonic matter, was of considerable importance.  Therefore, the NFW background potential was less suitable for our application.

In considering a suitable potential-density pair, characteristics that were important were:
\begin{itemize}
\item It must take into account baryonic matter
\item It must have a finite density in the center out to a scale radius
\item Declining density that drops more sharply out to a second scale radius
\item An even sharper decline in density beyond the second scale radius
\item It must be easily parametrized and written in analytic, closed form
\end{itemize}

For our simulations, the Stone-Ostriker profile (\citet{2015ApJ...806L..28S}), which is a three-parameter potential-density pair, was used.  It is essentially an analytic form of a finite, cored isothermal mass distribution.  In order to fully parametrize this profile so as to use it in our simulations, we needed to match the stellar and halo mass data from the Kulier et al simulations, to the velocity dispersion at the galaxy's effective radius.  In Kulier's simulations, a galaxy's effective radius $R_{e}$ and velocity dispersion $\sigma(R_e)$ at the effective radius are:
\begin{equation} \label{re}
R_{e} = 2.5 kpc\left(\frac{M_*}{10^{11}M_{\odot}}\right)^{0.73}(1+z)^{-0.98}
\end{equation}
\begin{equation} \label{sig}
\sigma(R_{e}) = 190km/s\left(\frac{M_{*}}{10^{11}M_{\odot}}\right)^{0.2}(1+z)^{0.47}
\end{equation}

Given the galaxy's stellar mass $M_{*}$ at redshift z from Kulier's simulations, we can calculate the effective radius and velocity dispersion.  We can then equate these values to the analytic expressions for $\sigma(R_{e})$ in the Stone-Ostriker profile.  Using a core radius $r_c$ = 100, we can solve for the halo radius $r_h$ using a simple recursive Newton method.  Whether $\sigma_{near}$ or $\sigma_{far}$ is used is determined by whether $R_e$ is less than or greater than $\sqrt{r_c r_h}$.

\subsubsection{Phase-space diffusion}
Weak encounters with background stars will let the SMBHs diffuse through phase space while they are orbiting within the gravitational potential of the galaxy. The diffusion can be expressed as change in velocity of an SMBH by $\Delta \vec{v}$ per unit time. We can split this change into a component along the direction of motion of the SMBH, and one perpendicular to that. Following \citet{Binney08}, the diffusion coefficients can be expressed as 
\begin{eqnarray}
D[\Delta v_\parallel] & = & -\frac{4\pi G^2\rho(r)M_\bullet\ln\Lambda}{\sigma^2}f(\chi),\label{eq:df}\\
D[(\Delta v_\parallel)^2] & = & \frac{4\sqrt{2}\pi G^2\rho(r)M_\bullet\ln\Lambda}{\sigma}\frac{f(\chi)}{\chi},\\
D[(\Delta \vec{v}_\bot)^2] & = & \frac{4\sqrt{2}\pi G^2\rho(r)M_\bullet\ln\Lambda}{\sigma}\left[\frac{\mbox{erf}(\chi)-f(\chi)}{\chi}\right],
\end{eqnarray} 
where $\Delta v_\parallel \equiv \Delta \vec{v}\cdot\vec{v}/v$ is the velocity change in direction of motion, and $\Delta \vec{v}_\bot \equiv \Delta \vec{v} - \Delta v_\parallel \cdot\vec{v}/v$ is the velocity change perpendicular to the direction of motion. Here, $M_\bullet$ is the mass of the black hole, and $\chi = \frac{v}{\sqrt{2}\sigma(r)}$. The function $f(\chi)$ is given by 
\begin{equation}
f(\chi) = \frac{1}{2\chi^2}\left(\mbox{erf}(\chi)-\frac{2\chi}{\sqrt{\pi}}\exp\left(-\chi^2\right)\right).
\end{equation}
We approximate the factor $\Lambda$ in the Coulomb logarithm as
\begin{equation}
\Lambda = \left(\frac{M_{NSC}}{M_\bullet}\right)\left(\frac{r}{r_h}\right).
\end{equation}
We can identify Eq.~\ref{eq:df} as the dynamical friction term, that is, if we assumed $D[(\Delta v_\parallel)^2]  = D[(\Delta \vec{v}_\bot)^2]  = 0$, we would get Chandrasekhar's dynamical friction formula. The second term introduces a variance of the friction term, and even allows the SBHs to be accelerated when the velocity of a SBH gets sufficiently small. The third term introduces a change in velocity perpendicular to the direction of motion of the SBH. It is a randomly oriented vector, and hence causes the SBHs to execute a random walk in phase space. The last two terms will establish that the SBHs are ultimately in energy equipartition with the background stars.
The velocity changes $\Delta v_\parallel$ and $\Delta\vec{v}_\bot$ per unit time $\Delta t$ can be computed with the above equations. Both changes are normal distributed, where the mean, $\mu$, and the variance, $\Sigma$, of the distributions are given by
\begin{eqnarray}
\mu_\parallel &=& D[\Delta v_\parallel]\Delta t,\\
\Sigma_\parallel &=& D[(\Delta v_\parallel)^2]\Delta t,\\
\mu_\bot &=& 0,\\
\Sigma_\bot &=& D[(\Delta \vec{v}_\bot)^2]\Delta t.
\end{eqnarray}
We compute the diffusion coefficients for each black hole at each time step, and modify its velocity on a Monte Carlo basis. For each time step we draw a random orientation before adding the perpendicular velocity change to the respective SBH. Hence, the SBH's modified velocity, $v_f$, is computed using
\begin{eqnarray}
\vec{v}_f &=& \vec{v}_0 + \Delta v_\parallel \hat{v}_\parallel + \Delta v_\bot \hat{v}_\bot,\\
\Delta v_\parallel &=& \mathcal{N}(\mu_\parallel, \Sigma_\parallel),\\
\Delta v_\bot &=& \mathcal{N}(\mu_\bot, \Sigma_\bot).
\end{eqnarray}
The change of energy, $\mbox{d}E_{BH}$, of the orbiting black hole due to phase-space diffusion is given back to the stellar background potential, with $\mbox{d}E = -\mbox{d}E_{BH}$. As a consequence of this energy transfer, inspiralling black holes will cause an expansion of the NSC. For this purpose we calculate the change in potential energy, $\mbox{d}W$, of the stellar system using
\begin{eqnarray}
E &=& T + W = \frac{1}{2}W,\\
\mbox{d}W &=& -2\,\mbox{d}E_{BH},
\end{eqnarray}
where we made use of the virial theorem $2T+W =0$. With this change in potential energy we can calculate a new radius for the stellar background potential at each integration step. For the Plummer sphere the new scale radius can be calculated as
\begin{equation}
a_{new} = a\left(1+\frac{32a\,\mbox{d}W}{3\pi GM_{NSC}^2}\right)^{-1}.
\end{equation}

\subsubsection{Gravitational wave recoils}
The code \textsc{AR-Chain} includes PN terms up to order 2.5. The SMBHs can therefore merge via gravitational wave emission. We include gravitational wave recoils following the prescription outlined in \citet{2015ApJ...799..178K}, which is based on the fitting formula by \citet{2012PhRvD..85h4015L}. To save computational time, we assume that a merger will be inevitable when the separation between two SBHs gets smaller than 1.0 Schwarzschild radii. At the moment of the merger, we assume that the spin vectors of the two SBHs are randomly aligned. This results in kick velocities of up to several thousand km\,s$^{-1}$, with a median kick of $\approx 290$\,km\,s$^{-1}$. Since our simulations focus on NSC with relatively low escape velocities, this implies that a majority of the merging SBHs escape from the NSCs. 

Black holes can also eject each other via strong three-body interactions. We remove SBHs from the simulations once they move beyond 1\,kpc from the NSC, assuming that it will take them more than a Hubble time to find their way back into the center of the host galaxy.


\subsection{Simulation setup}
Injection, merging, escape, effective radius, set of galaxies

Using the merger tree data, orbiting black holes were injected into its host galaxy at redshift z, at a distance from the galactic center of $R_{e}$ (Eqn. \ref{re}).  Their initial velocity was arbitrarily chosen to be circular, $v_c = \sqrt{M(R_e)/R_e}$, with $v_x$, $v_y$, and $v_z$ randomly chosen.

Note: we only inject the BHs into the simulation that have a $t_{fric}$ smaller than 100 times the Hubble time.


\subsection{Description of Simulations}
\begin{table*}
\centering
\caption{Galaxy characteristics}
\begin{tabular}{c| c c| c c| c c}
 & \multicolumn{2}{c}{$M_{gal}$ [$M_{\odot}$]} & 
\multicolumn{2}{c}{$M_{*}$ [$M_{\odot}$]} & 
\multicolumn{2}{c}{$M_{BH}$ [$M_{\odot}$]} \\
\hline
Galaxy & Init & Final & Init & Final & Init & Final \\
 \hline
A & $7.42\times10^{11}$ & $1.09\times10^{14}$  & $1.22\times10^{11}$ & $1.75\times10^{13}$ & $4.80\times10^{6}$ & $2.19\times10^{9}$\\
B & $1.11\times10^{12}$ & $3.41\times10^{13}$ & $1.80\mathrm{3}{11}$ & $5.58\times10^{12}$ & $1.94\times10^{8}$ & $1.46\times10^{9}$\\
C & $7.23\times10^{10}$ & $3.48\times10^{12}$ & $1.18\times10^{10}$ & $5.68\times10^{11}$ & $9.47\times10^{4}$ & $2.70\times10^{8}$\\
D & $3.92\times10^{10}$ & $1.34\times10^{12}$ & $6.47\times10^{9}$ & $2.20\times10^{11}$ & $9.95\times10^{5}$ & $8.32\times10^{7}$\\
\end{tabular}
\end{table*}





\section{Results}\label{sec:results}
\begin{table*}
\centering
\caption{SMBH statistics}
\begin{tabular}{c| c |c |c}
Galaxy & Infalling SMBHs & SMBHs with $t_{fric}<100\,t_H$ & Mergers \\
\hline
A & 206 & 19 & 4 \\
B & 29 & 12 & 4 \\
C & 3 & 3 & 2 \\
D & 4 & 4 & 2 \\
\end{tabular}
\end{table*}

\begin{figure*}[htbp]
\begin{center}
\includegraphics[width=0.45\textwidth]{plots/radius_A.png}
\includegraphics[width=0.45\textwidth]{plots/radius_B.png}\\
\includegraphics[width=0.45\textwidth]{plots/radius_C.png}
\includegraphics[width=0.45\textwidth]{plots/radius_D.png}\\
\caption{default}
\label{default2}
\end{center}
\end{figure*}

\begin{figure*}[htbp]
\begin{center}
\includegraphics[width=0.45\textwidth]{plots/orbiting_bh_mass_histogram_gal_1.png}
\includegraphics[width=0.45\textwidth]{plots/orbiting_bh_mass_histogram_gal_65.png}\\
\includegraphics[width=0.45\textwidth]{plots/orbiting_bh_mass_histogram_gal_187.png}
\includegraphics[width=0.45\textwidth]{plots/orbiting_bh_mass_histogram_gal_217.png}\\
\caption{default}
\label{default3}
\end{center}
\end{figure*}

\subsection{Mergers}
\begin{figure}[htbp]
\begin{center}
\includegraphics[width=0.45\textwidth]{plots/masses_ABCD.png}
\caption{default}
\label{default4}
\end{center}
\end{figure}




\noindent A: two mergers early on, and then two more at around 10-12 Gyr
B: 4 mergers across the final 8 Gyr
C: two mergers in the final Gyrs
D: ?


\subsection{Ejections}
No complete ejections, but kicks to higher energy orbits.
\noindent A: one in a three-body encoutner\\
B: 7-ish kicks, some in nearly 4-body encounters\\
C: none\\
D: ?\\

\subsection{Stalling SMBH binaries}
A: two periods of BH binary in center for > Gyr, up to 6 Gyr\\
B: two short periods of BHBs\\
C: 2 classic BHB periods of about 1 Gyr each\\
D: ?

\subsection{Stalling SMBHs in the halo}
\begin{figure*}[htbp]
\begin{center}
\includegraphics[width=0.45\textwidth]{plots/t_at_center_histogram_gal_1.png}
\includegraphics[width=0.45\textwidth]{plots/t_at_center_histogram_gal_65.png}\\
\includegraphics[width=0.45\textwidth]{plots/t_at_center_histogram_gal_187.png}
\includegraphics[width=0.45\textwidth]{plots/t_at_center_histogram_gal_217.png}\\
\caption{default}
\label{default5}
\end{center}
\end{figure*}

A: 95\% never make it into the central kpc\\
B: still about 90\% are in the halo\\
C: \\
D: ?



\section{Conclusions}\label{sec:conclusions}
What do we want to say?



\section*{Acknowledgements}

The authors would like to thank Andrea Kulier and Claire Lackner for providing merger tree data. AHWK acknowledges support by NASA through Hubble Fellowship grant HST-HF-51323.01-A awarded by the Space Telescope Science Institute, which is operated by the Association of Universities for Research in Astronomy, Inc., for NASA, under contract NAS 5-26555. 

\bibliographystyle{apj}
\bibliography{biblio}

%\begin{thebibliography}{}

%\bibitem[\protect\citeauthoryear{Aarseth}{2003}]{Aarseth03}
%Aarseth, S.~J., 2003, Gravitational N-Body Simulations (Cambridge University Press)


%\bibitem[\protect\citeauthoryear{K{\"u}pper et al.}{2011}]{Kupper11} 
%K{\"u}pper, A.~H.~W., Maschberger, T., Kroupa, P., Baumgardt, H., 2011, MNRAS, 417, 2300 

%\end{thebibliography}
\end{document}

